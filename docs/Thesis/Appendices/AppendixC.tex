% Appendix Template

\chapter{Message API} % Main appendix title

\label{AppendixC} % Change X to a consecutive letter; for referencing this appendix elsewhere, use \ref{AppendixX}

\lhead{Appendix C. \emph{Platform Message API}} % Change X to a consecutive letter; this is for the header on each page - perhaps a shortened title

The message are describes using the Google Protocol buffer format and the message definitions are stored in the \emph{proto} files which will be consumed by the code generators to generate the code in the desired programming language (one of C++, Python, Java, C\#). Although some of the primitive message definitions are imported from the Gazebo framework (which also uses Protocol buffers for message serialization), a lot of custom message types are defined for the Indriya framework. The Message API is listed in this Appendix.
\section{Message Type}\label{protocol-documentation}
\subsection{Gesture}\label{gesture.proto}
\subsubsection*{GestureDescription} Represents a gesture description
\begin{longtable}[l]{@{}llll@{}}
\toprule
Field & Type & Label & Description\tabularnewline
\midrule
\endhead
name & string & required & Name of the gesture\tabularnewline
type & GestureDescription.GestureType & required & Type of
gesture/motion\tabularnewline
active & bool & optional & Active status of gesture\tabularnewline
progress & int32 & optional & Gesture progress\tabularnewline
confidence & int32 & optional & Gesture recognition
confidence\tabularnewline
\bottomrule
\end{longtable}
\subsubsection*{GestureRecognitionModule} Represents a gesture recognition module
\begin{longtable}[l]{@{}llll@{}}
\toprule
Field & Type & Label & Description\tabularnewline
\midrule
\endhead
name & string & required & Module name\tabularnewline
params & Param & repeated & Module parameters\tabularnewline
motions & GestureDescription & repeated & List of gestures recognition
capabilities of this module\tabularnewline
\bottomrule
\end{longtable}
 \subsubsection*{GestureRecognitionModules} Represents a gesture recognition modules
\begin{longtable}[l]{@{}llll@{}}
\toprule
Field & Type & Label & Description\tabularnewline
\midrule
\endhead
modules & GestureRecognitionModule & repeated & Gesture recognition
module list\tabularnewline
\bottomrule
\end{longtable}
 \subsubsection*{GestureTrigger} Represents a gesture trigger
\begin{longtable}[l]{@{}llll@{}}
\toprule
Field & Type & Label & Description\tabularnewline
\midrule
\endhead
id & int32 & required & Trigger Identifier\tabularnewline
motion & GestureDescription & required & Recognized
gesture\tabularnewline
\bottomrule
\end{longtable}
\subsubsection*{GestureTriggers} Represents a list of gesture trigger
\begin{longtable}[l]{@{}llll@{}}
\toprule
Field & Type & Label & Description\tabularnewline
\midrule
\endhead
id & int32 & required & Identifier\tabularnewline
motion & GestureDescription & repeated & Recognized
gestures\tabularnewline
\bottomrule
\end{longtable}
 \subsubsection*{GestureDescription.GestureType} Represents gesture type
\begin{longtable}[l]{@{}lll@{}}
\toprule
Name & Number & Description\tabularnewline
\midrule
\endhead
None & 0 & No gesture\tabularnewline
Discrete & 1 & Discrete gesture\tabularnewline
Continuous & 2 & Continuous gesture\tabularnewline
\bottomrule
\end{longtable}
\subsection{Human}\label{human.proto}
\subsubsection*{Human} Represents information about a human
\begin{longtable}[l]{@{}llll@{}}
\toprule
Field & Type & Label & Description\tabularnewline
\midrule
\endhead
id & int32 & required & Unique identifier\tabularnewline
tracked & bool & required & True if tracked\tabularnewline
torso\_position & Vector3d & required & Position the
torso\tabularnewline
head\_position & Vector3d & required & Position of the
head\tabularnewline
orientation & Quaternion & required & Torso orientation\tabularnewline
\bottomrule
\end{longtable}
\subsubsection*{Humans} Represents a list of human
\begin{longtable}[l]{@{}llll@{}}
\toprule
Field & Type & Label & Description\tabularnewline
\midrule
\endhead
human & Human & repeated & List of human\tabularnewline
\bottomrule
\end{longtable}
\subsection{Joint Value}\label{jointux5fvalueux5fmap.proto}
\subsubsection*{JointValue} Represents the Joint value.
\begin{longtable}[l]{@{}llll@{}}
\toprule
Field & Type & Label & Description\tabularnewline
\midrule
\endhead
id & int32 & required & Joint Identifier\tabularnewline
value & double & required & Joint value\tabularnewline
\bottomrule
\end{longtable}
\subsubsection*{JointValueVector} Represents the list of joint values.
\begin{longtable}[l]{@{}llll@{}}
\toprule
Field & Type & Label & Description\tabularnewline
\midrule
\endhead
JointValues & JointValue & repeated & List of joints\tabularnewline
\bottomrule
\end{longtable}
\subsection{Kinect Body}\label{kinectux5fbody.proto}
\subsubsection*{KinectBodies} Represents a list of kinect body
\begin{longtable}[l]{@{}llll@{}}
\toprule
Field & Type & Label & Description\tabularnewline
\midrule
\endhead
Body & KinectBody & repeated & List of kinect bodies\tabularnewline
\bottomrule
\end{longtable}
\subsubsection*{KinectBody} Represents a body
\begin{longtable}[l]{@{}llll@{}}
\toprule
Field & Type & Label & Description\tabularnewline
\midrule
\endhead
TrackingId & int32 & required & Human Tracking Identifier\tabularnewline
IsTracked & bool & required & True if skeleton tracked\tabularnewline
JointCount & int32 & required & Number of joints in the
skeletoon\tabularnewline
Joints & KinectJoint & repeated & List of joints\tabularnewline
ClippedEdges & KinectBody.FrameEdges & optional & Occluded
edge\tabularnewline
HandLeftConfidence & KinectBody.TrackingConfidence & optional & Left
hand tracking confidence\tabularnewline
HandLeftState & KinectBody.HandState & optional & Left hand state
(open/closed/lasso)\tabularnewline
HandRightConfidence & KinectBody.TrackingConfidence & optional & Right
hand tracking confidence\tabularnewline
HandRightState & KinectBody.HandState & optional & Right hand state
(open/closed/lasso)\tabularnewline
IsRestricted & bool & optional & Restricted skeleton\tabularnewline
Lean & Vector2d & optional & Lean point\tabularnewline
LeanTrackingState & int32 & optional & Lean tracking
state\tabularnewline
\bottomrule
\end{longtable}
\subsubsection*{KinectBody.Activity} Represents human engagement
\begin{longtable}[l]{@{}lll@{}}
\toprule
Name & Number & Description\tabularnewline
\midrule
\endhead
EyeLeftClosed & 0 & Left eye closed.\tabularnewline
EyeRightClosed & 1 & Right eye closed.\tabularnewline
MouthOpen & 2 & Mouth open.\tabularnewline
MouthMoved & 3 & Mouth moved.\tabularnewline
LookingAway & 4 & Looking away.\tabularnewline
\bottomrule
\end{longtable}
\subsubsection*{KinectBody.Appearance} Represents the appearance of the human
\begin{longtable}[l]{@{}lll@{}}
\toprule
Name & Number & Description\tabularnewline
\midrule
\endhead
WearingGlasses & 0 & Wearing glasses.\tabularnewline
\bottomrule
\end{longtable}
\subsubsection*{KinectBody.DetectionResult} Represents the gesture recognition result
\begin{longtable}[l]{@{}lll@{}}
\toprule
Name & Number & Description\tabularnewline
\midrule
\endhead
Unknown & 0 & Undetermined detection.\tabularnewline
No & 1 & Not detected.\tabularnewline
Maybe & 2 & Maybe detected.\tabularnewline
Yes & 3 & Is detected.\tabularnewline
\bottomrule
\end{longtable}
\subsubsection*{KinectBody.Expression} The expression a body may have.
\begin{longtable}[l]{@{}lll@{}}
\toprule
Name & Number & Description\tabularnewline
\midrule
\endhead
Neutral & 0 & Neutral expression.\tabularnewline
Happy & 1 & Happy expression.\tabularnewline
\bottomrule
\end{longtable}
\subsubsection*{KinectBody.FrameEdges} Possible occlusion edges
\begin{longtable}[l]{@{}lll@{}}
\toprule
Name & Number & Description\tabularnewline
\midrule
\endhead
None & 0 & No frame edges.\tabularnewline
Right & 1 & Right frame edge.\tabularnewline
Left & 2 & Left frame edge.\tabularnewline
Top & 4 & Top frame edge.\tabularnewline
Bottom & 8 & Bottom frame edge.\tabularnewline
\bottomrule
\end{longtable}
\subsubsection*{KinectBody.HandState} The state of a hand of a body.
\begin{longtable}[l]{@{}lll@{}}
\toprule
Name & Number & Description\tabularnewline
\midrule
\endhead
HS\_Unknown & 0 & Undetermined hand state.\tabularnewline
HS\_NotTracked & 1 & Hand not tracked.\tabularnewline
HS\_Open & 2 & Open hand.\tabularnewline
HS\_Closed & 3 & Closed hand.\tabularnewline
HS\_Lasso & 4 & Lasso (pointer) hand.\tabularnewline
\bottomrule
\end{longtable}
\subsubsection*{KinectBody.TrackingConfidence} Represents tracking confidence
\begin{longtable}[l]{@{}lll@{}}
\toprule
Name & Number & Description\tabularnewline
\midrule
\endhead
Low & 0 & Low confidence.\tabularnewline
High & 1 & High confidence.\tabularnewline
\bottomrule
\end{longtable}
\subsection{Kinect Joint}\label{kinectux5fjoint.proto}
\subsubsection*{KinectJoint} Represents the Kinect skeleton joint data.
\begin{longtable}[l]{@{}llll@{}}
\toprule
Field & Type & Label & Description\tabularnewline
\midrule
\endhead
Type & KinectJoint.JointType & required & Type of skeleton
joint\tabularnewline
State & KinectJoint.TrackingState & required & Tracking state of the
joint\tabularnewline
Position & Vector3d & required & 3D position of the joint\tabularnewline
Orientation & Quaternion & required & 3D orientation of the
joint\tabularnewline
Angle & float & optional & Angle at the joint\tabularnewline
\bottomrule
\end{longtable}
\subsubsection*{KinectJoint.JointType} Represents the types of joints of a Body.
\begin{longtable}[l]{@{}lll@{}}
\toprule
Name & Number & Description\tabularnewline
\midrule
\endhead
SpineBase & 0 & Base of the spine.\tabularnewline
SpineMid & 1 & Middle of the spine.\tabularnewline
Neck & 2 & Neck.\tabularnewline
Head & 3 & Head.\tabularnewline
ShoulderLeft & 4 & Left shoulder.\tabularnewline
ElbowLeft & 5 & Left elbow.\tabularnewline
WristLeft & 6 & Left wrist.\tabularnewline
HandLeft & 7 & Left hand.\tabularnewline
ShoulderRight & 8 & Right shoulder.\tabularnewline
ElbowRight & 9 & Right elbow.\tabularnewline
WristRight & 10 & Right wrist.\tabularnewline
HandRight & 11 & Right hand.\tabularnewline
HipLeft & 12 & Left hip.\tabularnewline
KneeLeft & 13 & Left knee.\tabularnewline
AnkleLeft & 14 & Left ankle.\tabularnewline
FootLeft & 15 & Left foot.\tabularnewline
HipRight & 16 & Right hip.\tabularnewline
KneeRight & 17 & Right knee.\tabularnewline
AnkleRight & 18 & Right ankle.\tabularnewline
FootRight & 19 & Right foot.\tabularnewline
SpineShoulder & 20 & Between the shoulders on the spine.\tabularnewline
HandTipLeft & 21 & Tip of the left hand.\tabularnewline
ThumbLeft & 22 & Left thumb.\tabularnewline
HandTipRight & 23 & Tip of the right hand.\tabularnewline
ThumbRight & 24 & Right thumb.\tabularnewline
\bottomrule
\end{longtable}
\subsubsection*{KinectJoint.TrackingState} Represents the skeleton joint tracking state.
\begin{longtable}[l]{@{}lll@{}}
\toprule
Name & Number & Description\tabularnewline
\midrule
\endhead
NotTracked & 0 & / The joint data is not tracked and no data is known
about this joint.\tabularnewline
Inferred & 1 & / The joint data is inferred and confidence in the
position data is lower than\tabularnewline
Tracked & 2 & / if it were Tracked./ The joint data is being tracked and
the data can be trusted.\tabularnewline
\bottomrule
\end{longtable}
\subsection{Node}\label{node.proto}
\subsubsection*{Node} Represents information of a distributed node
\begin{longtable}[l]{@{}llll@{}}
\toprule
Field & Type & Label & Description\tabularnewline
\midrule
\endhead
name & string & required & Name of the node\tabularnewline
param & Param & repeated & Node parameters\tabularnewline
publisher & Publish & repeated & List of message
publishers\tabularnewline
subscriber & Subscribe & repeated & List of message
subscribers\tabularnewline
\bottomrule
\end{longtable}
\subsection{Param}\label{param.proto}
\subsubsection*{Param} Represents the Parameter Data.
\begin{longtable}[l]{@{}llll@{}}
\toprule
Field & Type & Label & Description\tabularnewline
\midrule
\endhead
key & string & required & Unique key (identifier) of the
parameter\tabularnewline
value & string & required & Parameter value\tabularnewline
dataType & string & required & Parameter Datatype (one of bool, int,
double, string, csv, file)\tabularnewline
\bottomrule
\end{longtable}
\subsubsection*{ParamList} Represents the list of Parameters.
\begin{longtable}[l]{@{}llll@{}}
\toprule
Field & Type & Label & Description\tabularnewline
\midrule
\endhead
param & Param & repeated &\tabularnewline
\bottomrule
\end{longtable}
\subsection{Publisher}\label{publish.proto}
\subsubsection*{Publish} Represents a message publisher
\begin{longtable}[l]{@{}llll@{}}
\toprule
Field & Type & Label & Description\tabularnewline
\midrule
\endhead
topic & string & required & Publisher topic\tabularnewline
msg\_type & string & required & Message type\tabularnewline
host & string & required & Host address\tabularnewline
port & uint32 & required & Host port\tabularnewline
\bottomrule
\end{longtable}
\subsection{Quaternion}\label{quaternion.proto}
\subsubsection*{Quaternion} Represents a Quaternion (axis,angle) =\textgreater{}(w = cos(angle/2), {[}x,y,z{]} = axis* sin(angle/2))
\begin{longtable}[l]{@{}llll@{}}
\toprule
Field & Type & Label & Description\tabularnewline
\midrule
\endhead
x & double & required & x = axis\_x * sin(angle/2)\tabularnewline
y & double & required & y = axis\_y * sin(angle/2)\tabularnewline
z & double & required & z = axis\_z * sin(angle/2)\tabularnewline
w & double & required & w = cos(angle/2)\tabularnewline
\bottomrule
\end{longtable}
\subsection{Robot Behavior}\label{robotux5fbehavior.proto}
\subsubsection*{BehaviorArguments} Represents the Robot behavior arguments.
\begin{longtable}[l]{@{}llll@{}}
\toprule
Field & Type & Label & Description\tabularnewline
\midrule
\endhead
name & string & required & Name of the argument\tabularnewline
value & string & required & Value of the argument\tabularnewline
place\_holder & bool & required & True if the value is dynamic otherwise
False\tabularnewline
type & string & required & Type of the argument (int, double,
string)\tabularnewline
\bottomrule
\end{longtable}
\subsubsection*{BehaviorDescription} Represents the Robot behavior description.
\begin{longtable}[l]{@{}llll@{}}
\toprule
Field & Type & Label & Description\tabularnewline
\midrule
\endhead
name & string & required & Name of the behavior\tabularnewline
function\_name & string & required & Remote procedure
name\tabularnewline
arg & BehaviorArguments & repeated & List of arguments required to
invoke the remote procedure\tabularnewline
type & BehaviorDescription.ExecutionType & required & Execution Type of
the behavior\tabularnewline
state & BehaviorDescription.ExecutionState & required & Execution status
of the behavior\tabularnewline
\bottomrule
\end{longtable}
\subsubsection*{RobotBehaviorModule} Represents the Robot behavior module.
\begin{longtable}[l]{@{}llll@{}}
\toprule
Field & Type & Label & Description\tabularnewline
\midrule
\endhead
name & string & required & Name of the behavior module\tabularnewline
param & Param & repeated & List of parameters of the behavior
module\tabularnewline
behaviors & BehaviorDescription & repeated & List of description of
supported behaviors\tabularnewline
responder & RobotBehaviorModule.RobotBehaviorResponder & optional &
Behavior module server information\tabularnewline
\bottomrule
\end{longtable}
\subsubsection*{RobotBehaviorModule.RobotBehaviorResponder} Represents the Behavior module's server information.
\begin{longtable}[l]{@{}llll@{}}
\toprule
Field & Type & Label & Description\tabularnewline
\midrule
\endhead
Host & string & required & Server host\tabularnewline
Port & int32 & required & Port\tabularnewline
\bottomrule
\end{longtable}
\subsubsection*{RobotBehaviorModules} Represents the List of Robot behavior modules.
\begin{longtable}[l]{@{}llll@{}}
\toprule
Field & Type & Label & Description\tabularnewline
\midrule
\endhead
modules & RobotBehaviorModule & repeated & List of behavior
modules\tabularnewline
\bottomrule
\end{longtable}
\subsubsection*{BehaviorDescription.ExecutionState} Represents the Execution state of the behavior.
\begin{longtable}[l]{@{}lll@{}}
\toprule
Name & Number & Description\tabularnewline
\midrule
\endhead
Idle & 0 & Behavior not running\tabularnewline
Running & 1 & Behavior is running\tabularnewline
Error & 2 & Behavior execution is in error state\tabularnewline
\bottomrule
\end{longtable}
\subsubsection*{BehaviorDescription.ExecutionType} Represents the Execution type of the behavior.
\begin{longtable}[l]{@{}lll@{}}
\toprule
Name & Number & Description\tabularnewline
\midrule
\endhead
Blocking & 0 & Execution is blocking\tabularnewline
NonBlocking & 1 & Execution is non-blocking\tabularnewline
\bottomrule
\end{longtable}
\subsection{Subscriber}\label{subscribe.proto}
\subsubsection*{Subscribe} Represents a message subscriber
\begin{longtable}[l]{@{}llll@{}}
\toprule
Field & Type & Label & Description\tabularnewline
\midrule
\endhead
topic & string & required & Subscribed topic\tabularnewline
host & string & required & Host address\tabularnewline
port & uint32 & required & Host port\tabularnewline
msg\_type & string & required & Message type\tabularnewline
latching & bool & optional & Latching enabled\tabularnewline
\bottomrule
\end{longtable}
\subsection{Vector 2d}\label{vector2d.proto}
\subsubsection*{Vector2d} Represents a 2D vector
\begin{longtable}[l]{@{}llll@{}}
\toprule
Field & Type & Label & Description\tabularnewline
\midrule
\endhead
x & double & required & X coordinate\tabularnewline
y & double & required & Y coordinate\tabularnewline
\bottomrule
\end{longtable}
\subsection{Vector 3d}\label{vector3d.proto}
\subsubsection*{Vector3d} Represents a 3D vector
\begin{longtable}[l]{@{}llll@{}}
\toprule
Field & Type & Label & Description\tabularnewline
\midrule
\endhead
x & double & required & x coordinate\tabularnewline
y & double & required & y coordinate\tabularnewline
z & double & required & z coordinate\tabularnewline
\bottomrule
\end{longtable}

 \#\# Scalar Value Types

\begin{longtable}[l]{@{}lllll@{}}
\toprule
.proto Type  & C++ Type & Java Type & Python Type\tabularnewline
\midrule
\endhead
double & double & double & float\tabularnewline
float & float & float & float\tabularnewline
int32 & int32 & int & int\tabularnewline
int64 & int64 & long & int/long\tabularnewline
uint32 & uint32 & int & int/long\tabularnewline
uint64 & uint64 & long & int/long\tabularnewline
sint32 & int32 & int & int\tabularnewline
sint64 & int64 & long & int/long\tabularnewline
fixed32 & uint32 & int & int\tabularnewline
fixed64 & uint64 & long & int/long\tabularnewline
sfixed32 & int32 & int & int\tabularnewline
sfixed64 & int64 & long & int/long\tabularnewline
bool & bool & boolean & boolean\tabularnewline
string & string & String & str/unicode\tabularnewline
bytes & string & ByteString & str\tabularnewline
\bottomrule
\end{longtable}
